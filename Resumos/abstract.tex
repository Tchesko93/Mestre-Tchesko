{\selectlanguage{english}
\begin{abstract}
\noindent In time series it is common to use machine learning to improve the performance of machines, in the process of training, validation and testing. Using the data was collected from SANEPAR of Curitiba - Paraná, this data was collected in the high neighborhood, and we will address some time series models collected in the systematic literature review, in what it addresses this review is in the years 2016 to 2022. Getting the models in the literature, to get a perspective of the best model made and which one fits the problem. For this dissertation the biggest problem of the data collected is to know what is the time that is consuming more water in the city and the neighborhood, in Curitiba there was even rodizio water making some neighborhoods without water for a certain period, to analyze minimally each point of the data use some models found and comparing such models to assume which is the best among the methods chosen, although the data from 2018 to 2019 is much more accurate and without many apparent anomalies, can be focused on the year 2020, which had the shortage of water or the beginning of it in the city. In addition to the models, it will also be used to complete and validate each one of them, classifying the best between the two, with a forecast horizon of 1, 10, 30, 60 days ahead, and seeing which one can be better in the short term and long term, to know which one will be the chosen model, it will be used metrics of errors known in the literature. The objective of this dissertation is to predict and try to change the future, so that the rotations do not occur again, with time to time having water.
\\
\\
\\

    
    \noindent \textbf{Keywords:} Forecasting, Water savings, Time series, Time series.
\end{abstract}
}
