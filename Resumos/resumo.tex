\begin{abstract}    
     \noindent Em séries temporais é comum usar aprendizado de máquina para melhorar o aproveitamento das máquinas, no processo de treinamento, validação e teste. Usando os dados foi coletado da SANEPAR de Curitiba - Paraná, esses dados foi coletado no \textit{Bairro Alto}, e abordaremos alguns modelos de série temporal coletado na revisão sistemática da literatura, no que lhe aborda essa revisão é nos anos de 2016 até 2022. Obtendo os modelos na literatura, para ter uma perspectiva do melhor modelo feito e qual se enquadra no problema. Para essa dissertação o problema maior dos dados coletados é saber qual é o horário que esta consumindo maior volume de água na cidade e o bairro, em Curitiba houve até rodizio de água tornando alguns bairros sem água por um certo período, para analisar minimamente cada ponto dos dados usar alguns modelos encontrado e comparando tais modelos para assumir qual é o melhor entre os métodos escolhidos, apesar dos dados de 2018 até 2019 ser bem mais preciso e sem muitas anomalias aparente, pode ser focado no ano de 2020, que houve a escassez de água ou o começo dela na cidade. Além dos modelos, vai ser usado também para completar e validar cada um deles, classificado como o melhor entre ambos, com o horizonte de previsão de 1, 10, 30, 60 dias a frente, e vendo qual deles pode ser melhor a curto prazo e a longo prazo, para saber quais vai ser o modelo escolhido, sera usado métricas de erros conhecidos na literatura. O objetivo dessa dissertação é prever e tentar mudar o futuro, para não ocorrer novamente os rodízios, com tempos em tempos tendo água.
    \\
    \\
    \\
    
    \noindent \textbf{Palavras-chave:} Previsão, Economia de água, Séries temporais, Série cronológica.
\end{abstract}

