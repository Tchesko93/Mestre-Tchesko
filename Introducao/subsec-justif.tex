\subsection{Justificativa da pesquisa} \label{subsec:justif}

No decorrer dessa dissertação ocorre da seguinte forma, para que possa ser previsto e para que seja evitado a efetiva falta d'água, e como pode ser solucionado esse problema para não voltar a acontecer.

\subsubsection{Contribuições} \label{subsubsec:Contribuição}

%%Ler tese do Matheus

A água como oxigênio têm uma importância significativa na vida humana, visando isso pode ser notado que sem ela eventualmente não existiria a humanidade, pois segundo \citeonline{walter} A água é a principal substância da vida. O corpo humano é composto de 48 a 54\% de água para pessoas adultas. Com o envelhecimento, a porcentagem de água no corpo humano diminui.

Tendo isso em mente a água que temos hoje pode ser um risco em acabar, como prova \citeonline{vasconcelos_2020} comenta isso no jornal Brasil fato, como os dados dessa pesquisa, vai até o mesmo ano da publicação desse artigo.

