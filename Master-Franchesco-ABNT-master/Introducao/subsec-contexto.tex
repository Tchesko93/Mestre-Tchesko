\subsection{Contexto da pesquisa} \label{subsec:contexto}
 Em séries temporais, o aprendizado de máquina é frequentemente utilizado para processamento de big data, com o conjunto de dados da SANEPAR em Curitiba - PR, na cidade há algum consumo e escassez de água, é necessário avaliar os dados para ter certeza do que está acontecendo, quando há escassez de água, e picos que ocorrem entre horas e dias.
 
 Dentre os modelos preditivos que serão apresentados em uma revisão sistemática, avaliar o melhor modelo que podemos utilizar e validar quando e como ocorre a escassez de água.
 Estas análises será em \textit{python} e o código final estará disponível em \textit{GitHub}.
 
 Link do repositório do \textit{GitHub} \href{https://github.com/Tchesko93/Mestre-Tchesko}{Repositório do Mestrado}
   
 Explorar o que são séries temporais e aprendizado de máquina, séries temporais são dados armazenados ao longo do tempo que permitem ao observador analisar anomalias nos dados. Em séries temporais, ordenar os dados por ano ou dia é fundamental e, se os dados forem catalogados aleatoriamente, os tomadores de decisão pode cometer erros e se perder ao analisar muitos dados.
   
      
\subsubsection{Motivação da pesquisa} \label{subsubsec:motivacao}
   %Escrever algo motivador 
    
    De acordo com \cite{vasconcelos_2020} Curitiba e região metropolitana enfrentou um rodízio com $36$ horas com água e $36$ horas sem abastecimento. A média geral dos reservatórios da região está em $27,96\%$ da capacidade. Assim em medida a isso essa pesquisa tem como a abordagem da falta de água, essa falta que pode ser vista como uma seca, em média nos anos anteriores de 2020 a chuva tem marcado a quantia de $1.704$ mm. \cite{vasconcelos_2020} Desde 2016, quando registrou 1.704 mm de chuva, Curitiba não atingiu mais a média anual de precipitação, que é de 1.490 mm, com base em dados da estação pluviométrica do Instituto Nacional de Meteorologia (Inmet).  Apesar de abaixo da média, o mínimo registrado desde então ocorreu em 2020, com 1.158 mm.
    
    Em mediano a essa motivação pode ser melhor interpretado os dados que a SANPEAR ofertor para prever e evitar a escasseia de água que foi registada, e a anomalia que foi detectado em 2020, com a volta da chuva os reservatórios teve aumento do nível.
    